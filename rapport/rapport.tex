
\documentclass[10pt]{scrartcl}
\usepackage[utf8]{inputenc}
\usepackage[frenchb]{babel}
\usepackage{lmodern}
\usepackage[T1]{fontenc}
\usepackage{hyperref}
\usepackage[margin=1in]{geometry}

\begin{document}
\title{Rapport du projet de Java EE}
\author{Maxence Ahlouche \and Maxime Arthaud \and Korantin Auguste \and Martin Carton}

\date{24 janvier 2013}
\maketitle

\section{Choix des technologies}
  Dès le début du projet, nous avions décidé d'utiliser quelque chose
  de plus évolué que ce qui nous a été présenté en cours. Notre choix
  s'est donc naturellement porté vers Spring. Toutefois, en surfant
  sur le World Wild Wed, nous avons trouvé un projet
  tout neuf qui nous a semblé particulièrement intéressant: Spring
  Boot\footnote{\url{https://github.com/spring-projects/spring-boot}}.

  Spring Boot est une surcouche de Spring, dont la philosophie est
  \emph{convention over configuration}. Cette approche convenait
  parfaitement à nos besoins, car nous n'avions aucune envie de
  configurer toutes les couches d'un projet JEE une par une. Ainsi,
  en une dizaine de minutes, il nous a été possible d'avoir une page
  web affichant des données d'une base de données. Un autre grand
  avantage de Spring Boot est qu'il permet de compiler toute
  l'application, ainsi qu'un serveur Tomcat, JPA et une base de
  données H2 (si besoin) en un seul fichier war, qu'il nous suffit
  de lancer avec Java pour avoir un serveur Web complet.

  Le principal inconvénient de cette approche est qu'il n'étais
  souvent pas trivial de modifier la configuration par défaut,
  d'autant plus que ce projet est encore jeune: le peu de
  documentation disponible était réparti entre plusieurs sites (leur
  Github, leur site, et leur ancien site plus maintenu), et était
  souvent incomplète. De plus, comme ce projet n'est pas encore
  utilisé suffisamment largement, StackOverflow ne nous a (pour une
  fois) pas été d'une grande aide.

  \paragraph{}
  Afin de gérer toutes nos dépendances, nous avons décidé d'utiliser
  Gradle, qui est une alternative à Maven. Bien qu'il soit également
  plus difficile de trouver de la documentation pour Gradle que pour
  Maven (car il est moins connu), ce choix nous a semblé le
  meilleur, car la syntaxe du fichier de configuration est bien plus
  user-friendly (nous avons réussi à faire un site Web fonctionnel
  en Java EE sans une seule ligne de XML).

  \paragraph{}
  Afin de représenter les vues, nous n'avons pas choisi les vues
  mais un système de templates nommé Thymeleaf. Bien que ce dernier
  ne soit certainement pas le meilleur qui soit (notamment parce
  qu'il préfère faire des includes à tout va plutôt que de faire de
  l'héritage de templates), il nous a permis de découvrir une autre
  manière de traiter le problème, qu'aucun d'entre nous n'avait vue
  auparavant, étant plutôt habitués au système de templates de Django
  (un autre framework Web en Python).

  \paragraph{}
  Concernant le choix du DBMS, notre choix s'est porté sur
  PostgreSQL, car à l'origine, il était prévu que soit mise à notre
  disposition une base de données sur un serveur de l'n7. Ne pouvant
  y accéder, nous avons mis en place notre propre serveur de base de
  données, qui était censé être temporaire, et qui aura finalement
  servi de serveur de «~production~».
\end{document}
