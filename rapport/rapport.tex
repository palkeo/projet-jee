\documentclass[10pt]{scrartcl}
\usepackage[utf8]{inputenc}
\usepackage[frenchb]{babel}
\usepackage{lmodern}
\usepackage[T1]{fontenc}
\usepackage{hyperref}
\usepackage[margin=1in]{geometry}

\begin{document}
\title{Rapport du projet de Java EE}
\author{Maxence Ahlouche \and Maxime Arthaud \and Korantin Auguste \and Martin Carton}

\date{24 janvier 2013}
\maketitle

\section{Introduction}
  Le choix du sujet étant libre, nous avons choisi de faire une plateforme pour
  se faire s'affronter des «~IA~» jouant à des jeux tour par tour\footnote{Le
  terme «~IA~» est un peu trompeur, nous entendons par là de simples programmes
  jouant aux jeux automatiquement.}.

  Les webmasters du site peuvent ajouter autant de jeux qu'ils le souhaitent
  (il suffit de programmer une simple classe jouant le rôle d'arbitre en Java,
  et un morceau de Javascript pour l'affichage).

  \paragraph{}
  Les utilisateurs quant à eux peuvent coder leurs «~IA~» dans n'importe quel
  langage. Les différentes «~IA~» et l'arbitre communiquent entre eux en JSON
  par socket. Les combats peuvent être lancés par les utilisateurs. Pour éviter
  de surcharger le serveur dans le cas où trop de combats se dérouleraient en
  même temps, les combats peuvent être lancés sur des machines distantes, qui
  récupèrent leurs tâches grâces à une queue JMS.

  \paragraph{}
  L'interface web permet de s'inscrire, de se connecter, d'uploader des «~IA~»,
  de les faire combattre contre celles des autres utilisateurs, de regarder les
  scores, de regarder le déroulement tour par tour d'un combat, etc.

\section{Choix des technologies}
  Dès le début du projet, nous avions décidé d'utiliser quelque chose
  de plus évolué que ce qui nous a été présenté en cours. Notre choix
  s'est donc naturellement porté vers Spring. Toutefois, en surfant
  sur le World Wild Wed, nous avons trouvé un projet tout neuf qui
  nous a semblé particulièrement intéressant: Spring
  Boot\footnote{\url{https://github.com/spring-projects/spring-boot}}.

  Spring Boot est une surcouche de Spring, dont la philosophie est
  \emph{convention over configuration}. Cette approche convenait
  parfaitement à nos besoins, car nous n'avions aucune envie de
  configurer toutes les couches d'un projet JEE une par une. Un autre
  grand avantage de Spring Boot est qu'il permet de compiler toute
  l'application, ainsi qu'un serveur Tomcat, JPA et une base de
  données H2 (si besoin) en un seul fichier war, qu'il nous suffit de
  lancer avec Java pour avoir un serveur Web complet.

  Le principal inconvénient de cette approche est qu'il n'était
  souvent pas trivial de modifier la configuration par défaut,
  d'autant plus que ce projet est encore jeune: le peu de
  documentation disponible était réparti entre plusieurs sites (leur
  Github, leur site, et leur ancien site plus maintenu), et était
  souvent incomplète. De plus, comme ce projet n'est pas encore
  utilisé suffisamment largement, StackOverflow ne nous a (pour une
  fois) pas été d'une grande aide.

  \paragraph{}
  Afin de gérer toutes nos dépendances, nous avons décidé d'utiliser
  Gradle, qui est une alternative à Maven. Ce choix nous a semblé le
  meilleur, car la syntaxe du fichier de configuration est bien plus
  user-friendly (nous avons réussi à faire un site Web fonctionnel en
  Java EE sans une seule ligne de XML $\backslash o/$).

  \paragraph{}
  Afin de représenter les vues, nous n'avons pas choisi les vues
  mais un système de templates nommé Thymeleaf. Bien que ce dernier
  ne soit certainement pas le meilleur qui soit (notamment parce
  qu'il préfère faire des includes à tout va plutôt que de faire de
  l'héritage de templates), il nous a permis de découvrir une autre
  manière de traiter le problème, qu'aucun d'entre nous n'avait vue
  auparavant, étant plutôt habitués au système de templates de Django
  (un autre framework Web en Python).

  \paragraph{}
  Concernant le choix du DBMS, notre choix s'est porté sur
  PostgreSQL, car l'un d'entre nous avait déjà un serveur configuré à
  disposition sur un serveur personnel.

\section{Backend : génération des matchs}
  L'interface web permet de lancer des matchs. Lors du lancement d'un
  match, une ligne est insérée dans la table «~match~» de la base de
  donnée, et un message contentant l'ID du match est ajouté dans la
  queue JMS.

  \paragraph{}
  Des «~workers~» jouent le rôle de consommateur de cette queue. Ils
  attendent un message, récupèrent les informations relatives au match
  dans la base de donnée, puis lancent le match. Pour cela, ils décompressent
  les archives des joueurs (placées dans un dossier qui serait par exemple partagé
  via NFS), et lancent le script bash $launch$ à la racine.

  \paragraph{}
  Nous avons utilisé le principe du classement Elo
  \footnote{\url{https://fr.wikipedia.org/wiki/Classement\_Elo}} provenant
  des échecs. Le principe est simple : un joueur part avec $1000$ points Elo.
  S'il gagne contre quelqu'un, ses points augmenteront en fonction de son
  score Elo et de celui de son adversaire. Par exemple, un match entre le
  premier et le dernier ne changera pratiquement pas les scores Elo. À l'inverse,
  si le dernier gagne, on pourra observer de grandes variations dans leurs scores.

\section{Problématique de sécurité}
  % TODO: palkeo

\end{document}
